\section{Introduction}
Technological advances in the past few decades have triggered an explosion in the collection of spatio-temporal data.  The increasing popularity of GPS devices and smartphones, along with the emergence of new disciplines such as the Internet of Things (IoT) and high-resolution Satellite/UAS imagery, has made it possible to collect vast amounts of data with spatial and temporal components.

In tandem, interest in extracting valuable information from such large databases has also grown.  Spatio-temporal queries about popular places or frequent events remain useful, but there has been growing interest in more complex patterns.  In particular, patterns that describe the group behavior of moving objects over significant periods.  Moving cluster \cite{kalnis_discovering_2005}, convoys \cite{jeung_discovery_2008}, flocks \cite{gudmundsson_computing_2006} and swarm patterns \cite{li_swarm_2010} reveal how entities move together over a minimum time interval.

Applications for this type of information are both diverse and intriguing, particularly when dealing with trajectory datasets \cite{jeung_trajectory_2011, huang_mining_2015}. Case studies span various domains, including transportation system management and urban planning \cite{di_lorenzo_allaboard_2016}, as well as ecology \cite{la_sorte_convergence_2016}. For example, \cite{turdukulov_visual_2014} explores the identification of complex motion patterns to discover similarities in tropical cyclone paths. Similarly, \cite{amor_persistence_2016} investigates eye movement trajectories to understand the strategies people use during visual searches. Additionally, \cite{holland_movements_1999} tracks the behavior of tiger sharks along the coasts of Hawaii to gain insight into their migration patterns.

One particular pattern of interest is the moving flock pattern, which captures how objects move within close proximity for a given time period. Closeness is defined by a disk of a specified radius within which the entities must remain. Since this disk can be positioned anywhere, detecting such patterns is a non-trivial problem. In fact, \cite{gudmundsson_computing_2006} highlights that finding flock patterns where the same entities stay together over time is an NP-hard problem. To address this, \cite{vieira_2009} proposed the BFE algorithm, the first approach capable of detecting flock patterns in polynomial time.

Despite the increasing availability of data, current state-of-the-art techniques for mining complex movement patterns still struggle with the performance demands of large-scale spatial data. This work introduces a scalable approach designed to detect moving flock patterns in very large trajectory databases. By leveraging emerging trends in distributed frameworks for spatial operations we aim to significantly improve the speed and efficiency of detecting these patterns.

\section{Related work}
The recent increased use of location-aware devices (such as GPS, smartphones, and RFID tags) has enabled the collection of vast amounts of data with spatial and temporal components.  Several studies have focused on discovering and analyzing these types of datasets \cite{leung_knowledge_2010, miller_geographic_2001}.  In this area, trajectory datasets have emerged as an interesting field where diverse kind of patterns can be identified \cite{zheng_computing_2011, vieira_spatio-temporal_2013}.  For instance, researchers have proposed techniques to discover spatial motion patterns such as moving clusters \cite{kalnis_discovering_2005}, convoys \cite{jeung_discovery_2008} and flocks \cite{benkert_reporting_2008, gudmundsson_computing_2006}.  Specifically, \cite{vieira_2009} introduced BFE (Basic Flock Evaluation), an innovative algorithm designed to efficiently identify moving flock patterns in polynomial time across large spatio-temporal datasets.

A flock pattern is defined as a group of entities that move together over a specified time period \cite{benkert_reporting_2008}. The applications of such patterns are broad and diverse. For instance, \cite{calderon_romero_mining_2011} identifies moving flock patterns in iceberg trajectories to analyze their movement behavior and their relationship with changes in ocean currents.

The BFE algorithm provides an initial approach for detecting flock patterns. It begins by identifying disks with a predefined diameter ($\varepsilon$) where moving entities are sufficiently close at specific time instants. This operation is computationally expensive due to the large number of points and time instances to be analyzed, with a complexity of $\mathcal{O}(2n^2)$ per time. Although the algorithm leverages a grid-based index and a stencil to accelerate this process, the overall complexity remains high.

Both \cite{calderon_romero_mining_2011} and \cite{turdukulov_visual_2014} adopt a frequent pattern mining approach to enhance performance when combining disks across time instants. Similarly, \cite{tanaka_improved_2016} utilize plane sweeping techniques, binary signatures, and inverted indexes to further accelerate this process. However, these methods retain the core strategy of BFE for detecting disks at each time instant.

In contrast, \cite{arimura_finding_2014} and \cite{geng_enumeration_2014} employ depth-first algorithms to analyze the time intervals of individual trajectories and report maximal duration flocks. However, these methods are less effective for dense datasets or those that involve large numbers of entities per time step, as they struggle to scale efficiently in such conditions.

Given the high computational demands of flock pattern detection, it is not surprising that parallelism has been employed to improve performance. For example, \cite{fort_parallel_2014} use extreme and intersection sets to report maximal, longest, and largest flocks on GPUs, albeit with limitations imposed by the GPU's memory model.

Despite the increasing adoption of cluster computing frameworks, particularly those with spatial data capabilities \cite{eldawy_spatialhadoop_2014, yu_demonstration_2016, pellechia_geomesa_2015, xie_simba_2016}, significant advancements in this area remain limited. To the best of our knowledge, this work is the first to explore the detection of moving flock patterns in a scalable approach.
